\chapter{AI for Autonomous Robots}
Locations next to door have increased belief. Posterior belief, after measurement.
Robot moves to the right and the posterior changes.
The new posterior and the old one are combined to produce a output posterior that magically has a high probability for the 2nd door and low elsewhere, which is what we want. This is the core of localization.
\lstinputlisting[language=Python]{./Online-Courses/Udacity/AI-Robots/prob1.py}
\lstinputlisting[language=Python]{./Online-Courses/Udacity/AI-Robots/prob2.py}
\lstinputlisting[language=Python]{./Online-Courses/Udacity/AI-Robots/prob3.py}
\lstinputlisting[language=Python]{./Online-Courses/Udacity/AI-Robots/prob4.py}
\lstinputlisting[language=Python]{./Online-Courses/Udacity/AI-Robots/prob5.py}
\lstinputlisting[language=Python]{./Online-Courses/Udacity/AI-Robots/prob6.py}
